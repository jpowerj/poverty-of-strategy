\documentclass{article}
\usepackage{amsmath}
\usepackage{amssymb}
\usepackage{mathtools}
\usepackage{fullpage}
\usepackage[T1]{fontenc}
\usepackage{lmodern}
\usepackage{setspace}
\usepackage{hyperref}

\newcommand{\term}[1]{\textbf{#1}}
\newcommand{\eg}[0]{\textit{e.g.}}

\title{Snippets}

\begin{document}

\maketitle

\section{Ockham's Razor}

Ockham's Razor, as a technique for deciding between competing theories, is a \term{heuristic} rather than a principled inference technique like Bayes' rule.

\section{Cognitive Biases and Statistics, or, Why Humans are Terrible at Probability}

\subsection{Base Rate Fallacy}

Here we run into the first example of how ignoring the principles of Bayesian reasoning can lead us astray. The \term{base rate fallacy} is a systematic cognitive bias humans tend to fall for where important \textit{general} information (\eg{} the titular base rate) gets 

\section{Optimization Prob}

The agent's optimization problem:

\begin{align}
\text{minimize } & Lx_i + z_i \\
\text{subject to } & (p-pa)x_i + [p - (pa + L)]y_i + z_i \geq pb \\
 & pax_i + pay_i \leq p\omega_i \\
 & Lx_i + z_i \leq 1 \\
 & x_i > 0 \wedge y_i > 0 \wedge z_i > 0
\end{align}

Given this \textit{individual} optimization problem, we can analyze outcomes in the economy by defining corresponding \textit{aggregate} quantities
\begin{align*}
x = \sum_{i=1}^N x_i, \; y = \sum_{i=1}^N y_i, \; z = \sum_{i=1}^N z_i
\end{align*}

And a price of corn $p$ represents an equilibrium in this model if, after every agent chooses their production vector $\langle x_i, y_i, z_i\rangle$, the aggregate quantities $x$, $y$, and $z$ satisfy
\begin{align}
(1-a)(x+y) &\geq Nb \\
Ly &= z \\
a(x + y) &\leq \omega
\end{align} 



\end{document}